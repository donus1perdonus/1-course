\documentclass{article}
\usepackage[T2A]{fontenc} % Указываем кодировку T2A для поддержки кириллицы
\usepackage[utf8]{inputenc} % Указываем UTF-8 кодировку для ввода символов
\usepackage[russian]{babel} % Подключаем пакет babel для русификации
\begin{document}
Контрольная работа: 1

Группа: 1

Вариант: 1

\begin{enumerate}
\item Прямая: \[y = 10x - 4\] Параллельна касательной к графику функции: \[g = 3x^2 + 8x - 4\]
 Найдите абсциссу точки касания.

\item Задание. Найти производную от полинома.
\[ f(x) = 5x^5 + 5x^4 - 3x^2 - 4x \]

\item Задание. Найти наибольшее и наименьшее значения функции на промежутке [-2, 0]. 

\[ f(x) = 10x^3 + 9x^2 - 6 \]

\item Прямая: \[y = 10x - 4\] Параллельна касательной к графику функции: \[g = 3x^2 + 8x - 4\]
 Найдите абсциссу точки касания.

\item Задание. Найти производную от полинома.
\[ f(x) = 5x^5 + 5x^4 - 3x^2 - 4x \]

\item Задание. Найти наибольшее и наименьшее значения функции на промежутке [-6, 1]. 

\[ f(x) = 8x^3 + 10x^2 - 5 \]

\item Задание. Найти значение производной в точке 8.
\[ f(x) = 7x^5 + 3x^4 + 4x^3 - 9x + 3 \]

\item Задание. Найти производную от полинома.
\[ f(x) = 5x^5 - 2x^4x^3 + 4x^2 - 10 \]

\item Прямая: \[y = 10x - 4\] Параллельна касательной к графику функции: \[g = 3x^2 + 8x - 4\]
 Найдите абсциссу точки касания.

\item Задание. Найти производную от полинома.
\[ f(x) = 5x^5 + 5x^4 - 3x^2 - 4x \]

\item Задание. Найти наибольшее и наименьшее значения функции на промежутке [-6, 1]. 

\[ f(x) = 8x^3 + 10x^2 - 5 \]

\end{enumerate}
\end{document}
