
\documentclass[a4paper, 14pt] {extarticle} 
\usepackage [utf8] {inputenc} 
\usepackage [T2A] {fontenc} 
\usepackage [russian] {babel}

\begin{document}

\section*{Пояснительная записка}

А. Приложение позволяет заполнить динамический массив структур.

Б. Пользователь вводит строку (\texttt{char*}) через терминал.

В. Пользователь вводит сообщения с клавиатуры через терминал. Ввод прекращается, когда пользователь вводит слово "Стоп". Приложение записывает сообщения во временный файл \texttt{2\_tmp.txt}. Каждому сообщению присваивается свой идентификационный номер (\texttt{id}), и сообщается его размер в байтах.

Г.Пример корректной работы приложения:
\begin{verbatim}
Введите сообщение: Артём
Введите сообщение: Кристина
Введите сообщение: Дима
0 Артём 6
1 Кристина 9
2 Дима 5
\end{verbatim}

Д. Так как входными данными является массив символов, то аномальные типы данных не являются помехой.

Е. Разработано 2 модуля: модуль \texttt{2.с} представляет собой функцию main, модуль \texttt{2lib.с} является модулем, содержащим в себе основные функции, используемые в main.

Ж. Функция для очистки массива символов - \texttt{freestr}, функция для посимвольного копирования строки - \texttt{copystr}, функция для передачи информации в логи - \texttt{logtoFile}.

\end{document}

